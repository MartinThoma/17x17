\documentclass[a4paper,11pt]{scrartcl}
\usepackage[ngerman]{babel} 
\usepackage[utf8]{inputenc}
\usepackage{amssymb,amsmath}
\usepackage{dsfont}
\usepackage{geometry}
\usepackage{graphicx}

\geometry{a4paper,left=18mm,right=18mm, top=1cm, bottom=2cm} 
\setlength{\columnsep}{120pt}


\setcounter{secnumdepth}{2}
\setcounter{tocdepth}{2}

\begin{document}
 \title{17x17 Problem}
 \author{Martin Thoma}

 \setcounter{section}{1}
 \section*{Das 17x17 Problem}
 Das 17x17 Problem bezieht sich auf die Färbung mit vier Farben.
 \subsection{Zahlen}
   \begin{itemize}
     \item Anzahl der möglichen Färbungen: $4^{(17^2)} = 4^{289} \approx 9,89 \cdot 10^{173}$
     \item Anzahl der 4er Permutationen: $\binom{17}{4} = 2380$
     \item Anzahl der 5er Permutationen: $\binom{17}{5} = 6188$
     \item Anzahl aller Kombinationsmöglichkeiten: $\binom{2380+6188}{17} = \binom{8568}{17} \approx 2,00 \cdot 10^{52}$
     \item Wenn es 1 Lösung gibt, gibt es mindestens: $17! \cdot 17! \approx {3,56 \cdot 10^{14}}^2 \approx 1,27 \cdot 10^{29}$
   \end{itemize}

\end{document}
